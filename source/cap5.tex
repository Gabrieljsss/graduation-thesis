\paragraph{} CERN provides a distinctive environment for teams to experiment, adapt, and advance technologies and processes. In this setting, the Glance team successfully overhauled the Membership system with a new architectural design. This initiative aimed to enhance system usability and incorporate previously unavailable features, stemming from the constraints of the Fence-based architecture. By re-evaluating Membership and Authorship requirements, it became evident that the system could be better tailored to support real-world processes. This realization underscored the need to transition to a technology sufficiently flexible to allow integrating the requested new functionalities. The development of the Authorship system, serving as a proof of concept for the Hexagonal Architecture with a decoupled backend API and frontend, allowed the team to standardize the Hexagonal Architectural pattern. This pattern not only established a standard approach for software development but also maintained the flexibility and modularity necessary for implementing complex and unique features such as the Search Library, implemented to fulfill the gap left by the Fence Super Search, unblocking the Membership refactor project. Following the establishment of the technology stack, Membership System Version 2 was launched by mid-2021. Subsequent to its release, two additional systems were developed using the same architectural principles, further validating the new architecture's robustness.

\paragraph{} Even though the new architecture facilitated the integration of novel features and yielded quantifiable productivity enhancements, areas for further improvement exist. The adoption of the Slim Framework 4 coupled with the application-specific middlewares in place of FRAPI could potentially boost performance. This transition would enable applications dependent on FRAPI, such as the Membership system, to perform configurations directly within the code rather than utilizing JSON configuration files, thereby avoiding file-read operations. This also gives developers the freedom to only install the middlewares that are actually going to be used by their applications. A great effort to remove FRAPI's CERN-specific middlewares to standalone bundles, particularly those managing authentication and authorization, is currently a priority for the Glance team allowing applications to combine these middlewares with the Slim Framework without FRAPI. Another aspect of possible enhancement on the backend is the implementation of more caching solutions. Given the frequent querying of numerous entities throughout the day, in-memory caching could significantly speed up common searches. CERN's proprietary server infrastructure offers developers a wide array of hosting options, further facilitating these improvements. On the frontend, transitioning the script section of Vue's SFCs from JavaScript to TypeScript would also be beneficial. TypeScript offers advantages over JavaScript, including static typing, class-based object-oriented programming, and compile-time error checking, which collectively enhance code reliability and maintainability. Another proposed enhancement involves refining the search interfaces to intuitively deduce the Search Field based on user input. For instance, if a user begins typing ``10/01...'', the system could automatically recommend date-related Search Fields, thereby streamlining the user interface from three inputs to a single, more intuitive input. This modification aligns with overarching principles advocating for simplified user interfaces. Lastly, upgrading from Vue 2 to Vue 3 would facilitate the adoption of the new Composition API, which provides a more flexible and modular approach to composing component logic. This upgrade would leverage the Composition API's advantages, including improved TypeScript support, enhanced reusability, and better code organization, thereby elevating the frontend development experience.

