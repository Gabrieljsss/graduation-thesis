\section{Theme} \paragraph{} This work presents the refactoring of the LHCb (Large Hadron Collider beauty) Membership Web Application, used by one of the four primary experiments at CERN (The European Organization for Nuclear Research) to manage its collaboration members and institutions. The project focused on transitioning the system from a tightly integrated monolithic structure, containing three different but loosely connected applications, to a Modular Monolith Web Application. The redesign emphasized the ``separation of concerns" principle, a programming approach that divides an application into distinct units with minimal overlap in functionality, achieved through modularization, encapsulation, and arrangement in software layers \cite{SAP_ABAP_Doc}. This principle guided the creation of clear boundaries between frontend and backend modules, which now interact through a strict API (Application Programming Interface) contract.


\section{Scope}

\paragraph{}  CERN, is a major international center for scientific research in particle physics. It operates the largest particle physics laboratory in the world, where physicists and engineers investigate the fundamental components and forces of the universe. The facility is known for conducting four large experiments: ATLAS (A Toroidal LHC ApparatuS), CMS (Compact Muon Solenoid), ALICE (A Large Ion Collider Experiment), and LHCb, each designed to study different aspects of particle physics. Universities and research centers participate in the experiments at CERN by contributing to various aspects including the design, construction, and operation of experimental equipment, as well as the analysis of data. Researchers, professors, and students from institutions globally are involved in these experiments, offering their expertise and conducting research projects aligned with CERN's objective. 

\paragraph{}  One significant contribution from UFRJ is Glance, a data retrieval tool developed during the UFRJ-CERN collaboration for the ATLAS experiment in 2003. Glance functions as a web application and serves as an intermediary layer between the end-user and the database. This design enables users to insert and retrieve information from the database without requiring SQL (Structured Query Language) knowledge, simplifying data access and manipulation. This technology, by 2013, evolved into the FENCE (Frontend Engine for Glance) Framework, which is an object-oriented PHP library that powers web applications configured through JSON (JavaScript Object Notation) configuration files, allowing some basic input validation and interface customization. The Membership System Version 1, created with FENCE, is designed to manage participants and their affiliations in the LHCb collaboration. This system handles tasks such as member employment management, institute cooperation agreements, data access control, special role assignments (appointments), and automated authorship list generation. The Membership System became an integral component of collaboration management. However, an increase in new requirements highlighted the limitations of FENCE's configuration-file-based architecture. Similar challenges were noted in other FENCE-based systems within the ALICE and ATLAS collaborations, prompting a collective initiative to seek an alternative solution by the begging of 2020.

\paragraph{}The author was based in Geneva from January 2020 to March 2022 under the guidance of Professor José Seixas from LPS COPPE/Poli/UFRJ (Laboratório de Processamento de Sinais) and CERN staff Gloria Corti and Joel Closier. The project timeline started with authorship development at the beginning of 2020, followed by the search tool, and finally the membership refactor from mid-2020 to late 2022.

\section{Justification}

\paragraph{} In scenarios where the existing codebase proves too restrictive or antiquated to effectively adapt to current requirements or integrate new features, the option to rewrite the software is often contemplated, as discussed in ``Refactoring: Improving the Design of Existing Code" by Martin Fowler \cite{fowler2019refactoring}. This approach enables addressing the limitations of the past and harnessing creativity in software development. Instances where rewriting may be necessary include situations where adding new features is impossible without a complete overhaul, onboarding new developers becomes overly complex, the existing platform is no longer supported, or there's a need to support a significantly increased user base. Rewriting allows for the adoption of modern interfaces, technologies, and can offer a more efficient system model based on a deeper understanding of the product's domain. However, it is important to recognize that rewrites can be time-consuming, risk introducing new bugs, and require maintaining both the old and new systems simultaneously \cite{kim2014refactoring}. Additionally, FENCE had been extensively used and expanded during its first 6 years, providing critical tools such as The GlanceSearchInterface enabling users to perform, replicate, and share data searches, presenting results in a customizable tabular format with options for CSV and PDF exports with the SuperSearch class expanding this functionality, allowing for complex searches with multiple parameters, organized logically in a graphical interface for intuitive user interaction and precise search criteria formation.

\paragraph{} Using any framework in application development has intrinsic limitations and challenges. One of the main issues is that frameworks, while customizable, often impose design limitations and restrictions. This means that developers need to adapt their projects to fit within the constraints of the framework, which might not always align with the business requirements. FENCE apps were structured in monolithic GitLab repositories containing all systems, which in LHCb were 3: The Membership, The LHCb equipment Management System (LBEMS) and the LHCb Cables. Meaning that releases had to consider changes in all of them, even if the goal is to deploy changes only in one. The lack of documentation, associated with the high turnover rate in the Glance Team, made changes in the framework itself more risky as unknown side effects could arise in any of the 20 FENCE powered apps. Consequently, a backlog of issues accumulated over the years and the newly onboarded developers had an increasingly difficult to modify the framework code.

\paragraph{} Another issue encountered with FENCE was the entanglement of its model-view-controller inspired layers, often consolidated within a single file. This integration led to challenges in testing applications, as changes became unpredictable and software maintainability was compromised. Due to the high coupling between different classes, such as HTTPS request controllers, database manager classes, and frontend callbacks, input validation was dispersed across these components. Consequently, this could lead to inconsistent outcomes, such as false positives, where valid data is incorrectly rejected by the database, or false negatives, where invalid data passes initial layers but is caught by the database. Moreover, the absence of a clear and centralized location for business logic validation often resulted in exceptions being thrown by the database, producing errors that were not meaningful to users or developers. A typical example is the occurrence of a unique constraint violation error when trying to insert a duplicate entry into a database table. Such an error message, while accurate, does not provide context or guidance for resolving the issue.
 

\paragraph{} Furthermore, a notable drawback of FENCE was its reliance on server-side generation of frontend interfaces. Generating interfaces on the server side can lead to disadvantages, such as increased response times due to the need for server-side processing before content delivery. This approach can also limit dynamic interaction on the client side, as each user interaction might require server communication, leading to less responsive and interactive user experiences. The fact that LHCb is a geo-distributed scientific collaboration meant that a considerable percentage of the user base accesses the systems from different countries and being a self-hosted system, it becomes necessary to provide ways to reduce latency.

\paragraph{} Finally, there was an increasing need within the LHCb Membership system to develop workflows that more accurately mirrored real-life processes, moving beyond simple CRUD operations. This shift was accompanied by a growing demand for more sophisticated user interfaces. However, these evolving requirements began to highlight the limitations of the existing architecture, as many of the new user requests were not feasible within its current framework. This situation amplified the need to explore alternative solutions that could offer the same fundamental functionalities as FENCE, but with greater flexibility to accommodate these more complex requirements.






%tests

%onboarding and lack of documentation

%incresasing framework backlog

%ddd


%\paragraph{}Apresentar o porquê do tema ser interessante de ser estudado. Cuidado, não é a motivação particular. Devem ser apresentadas razões para que %lguém deva se interessar no assunto, e não quais foram suas razões particulares que motivaram você a estudá-lo (tamanho do texto: livre).


\section{Objectives}

This project aims at rewriting the LHCb Membership System in a more modern software architecture and technology stack focusing on flexibility and community support. The key objectives were
\begin{itemize}
    \item Revalidate existing software requirements and collect new ones;
    \item Implement a search solution to replace FENCE's Super Search;
    \item Define and consolidate a new software architecture;
    \item Implement a proof of concept to validate the proposed changes;
    \item Implement the LHCb Membership Version 2.
\end{itemize}

%\paragraph{}Informar qual é o objetivo geral do trabalho, isto é, aquilo que deve ser atendido e que corresponde ao indicador inequívoco do sucesso do seu trabalho. Pode acontecer que venha a existir um conjunto de objetivos específicos, que complementam o objetivo geral (tamanho do texto: livre, mas cuidado para não fazer uma literatura romanceada, afinal esta seção trata dos objetivos).


\section{Methodology}

\paragraph{} This process started with the study of the existing RFC (request for comments) documents created by the team with the goal to better understand the new architectural proposal. After that, a literature review was carried out to map the current industry standards for frontend development and to define some aspects of the backend which were not covered by the RFCs nor implemented in Frapi: a new module created by another Glance Developer to help application to set up REST APIs (Representational State Transfer Application Programming Interfaces) to the new applications' backend. One of these aspects was data retrieval and persistence, which could be powered by and ORM tool or raw SQL queries, among other aspects. 

\paragraph{} Next, a requirement gathering process started to determine the necessary components for the chosen project to be the new architecture's proof of concept in LHCb: the Authorship System. This process included customer interviews with the LHCb Editorial Board Chairperson, reviewing relevant documents like the LHCb Constitution and internal records, and analyzing comments in the issue tracker application. Additionally, a review of the existing authorship system's production version was conducted to understand the implemented algorithm. This approach aimed to gather a clear and complete picture of the necessary features and functionalities for the new system.

\paragraph{} Once the new Authorship system was deployed in production, the Membership Refactor started. At this moment, other system had already been migrated to the new stack, so the new architecture was more consolidated. Another requirements gathering round occurred, revealing the necessity of new tools. The first was a search tool similar to what already existed in FENCE and the other a workflow tracking tool, to monitor the state of internal processes. The Membership Version 2 development was carried out in parallel to the production version instead of an incremental development. Beta users were constantly invited to test new features and give feedback of how the V2 compared to the FENCE-based version, and these comments / requests guided the development.

\section{Description}

\paragraph{} This work is organized in three parts. Chapter two extends the context, explaining how the international collaboration between CERN and UFRJ is structured, as well as the challenges that arise from it. Then, FENCE's most notorious drawbacks are explored to enrich the justification for a new software architecture solution. This chapter is focused on the software business rules presenting the search tool problem, and then the Membership system requirements and goals for the refactor.
\paragraph{} Chapter three discusses the implementation. Here the tools used to accomplish the goals set in the chapter before are described as well as the strategy followed to reach those goals. Finally, in the conclusion, the results from architectural changes will be presented using metrics to compute the productivity gain from the changes deployed. It will also present suggestions for future developments. 
