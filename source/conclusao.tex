Entende-se que é com ceticismo que deve-se ewncarar a presunção da ciência relativamente a sua objetividade e ao seu caráter definitivo. Tem-se dificuldade em aceitar que os resultados científicos, principalmente em neurociência computacional, sejam algo mais que aproximações provisórias para serem utilizadas por uns tempos e abandonadas logo que surjam modelos melhores. No entanto, o ceticismo relativo ao atual alcance da ciência, especialmente no que diz respeito à mente, não envolve menos entusiasmo na tentativa de melhorar as aproximações provisórias.


A redução das emoções pode constituir uma fonte de comportamento irracional. Essa ligação aparentemente ilógica entre ausência de emoções e comportamento anômalo pode ensinar-nos muito sobre o mecanismo biológico da razão.


\paragraph{}Conforme descrito por Damásio \cite{Damasio96} deve-se avaliar com ceticismo na noção dualista com a qual Decartes separa a mente do cérebro e do corpo, bem como as variantes modernas desta noção. Por exemplo, a idéia que a mente e cérebro estão relacionados mas apenas no sentido de que a mente é o programa de \textit{software} que é executada numa parte do \textit{hardware} chamado cérebro; ou que o cérebro e corpo estão relacionados, mas apenas no sentido de o primeiro não conseguir sobreviver sem a manutenção que o segundo lhe oferece. A afirmação de Descartes, "Penso, logo existo", sugere que pensar e ter consciência de pensar são os verdadeiros substratos das existência. E como sabemos que Descartes via o ato de pensar como uma atividade separada do corpo, essa afirmação celebra a separação da mente, a coisa pensante, do corpo não pensante, o qual tem extensão e partes mecânicas.
No entanto, antes do aparecimento da humanidade, os seres já eram serem. Num dado ponto da evolução, surgiu uma consciência elementar. Com essa consciência elementar apareceu uma mente simples; com uma maior complexidade da mente veio a possibilidade de pensar e, mais tarse ainda, de usar linguagens para comunicar e melhor organizar os pensamentos. Existimos e depois pensamos e só pensamos na medida que existimos, visto o pensamento ser, na verdade, causado por estruturas e operações do ser.
